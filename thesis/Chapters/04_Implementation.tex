\chapter{Implementation}

% \section{Overview}

In this section, we outline the general structure and design choices that went into implementing the various methods and models. 
While we will not go into the minutiae of the source code, it can be found at the GitHub repository, \url{https://github.com/sorenhartmann/Speciale}.

We primarily use three different frameworks for the implementation, PyTorch, PyTorch Lightning, and Hydra.
PyTorch provides the base deep learning methods underpinning the deep learning models. 
PyTorch Lightning provides many additional features, such as dealing with different computational devices such as GPU or TPU, data splitting, an extensive callback API, and more.
Importantly, PyTorch Lightning also encapsulates the inference process through the use of LightningModules.
In implementing the various methods, a particular goal has been to separate the various components, such as models, data, and inference, as much as possible.  
This makes it possible to effectively compare different methods of inference across different models and data.
The overall architecture is based around composition and can be seen in \cref{fig:sw-arch}.
\begin{figure}[htbp]
    \centering
    % spellchecker: disable

\begin{tikzpicture}[
    class/.style={rounded corners=0.3cm,minimum height=1cm,fill=blue!50,draw=black},
    back group/.style={rounded corners, draw=black, 
                     dashed, inner xsep=0.2cm, inner ysep=0.3cm, 
                     anchor=west},
    node distance = 1cm, 
    auto
    ]
    \node[class,fill=brightgreen!50] (sgd-module) at (2, 4.75) {SGDInference};
    \node[class,right=0.2cm of sgd-module.east,fill=brightgreen!50] (vi-module) {VariationalInference};
    \node[class,right=0.2cm of vi-module.east,fill=brightgreen!50] (mcmc-module) {MCMCInference};
    \node[class,above=1.2cm of mcmc-module] (sampler) {Sampler};
    \begin{scope}[on background layer]
        \node (inf-modules) [back group] [fit=(sgd-module)(vi-module)(mcmc-module),label={below right:InferenceModule}]{};
    \end{scope}
    \node[class, left=of inf-modules] (data-module) {DataModule};
    \node[class, above=1cm of data-module] (dataset) {Dataset};
    \node[class, align=center, above=2cm of inf-modules] (conv) {Bayesian-\\ConversionConfig};
    \node[class,left=1cm of conv] (model) {Model};
    \path (data-module) -- (mcmc-module) node[midway] (center) {};
    \node[class, below=2.5cm of center, fill=grey] (trainer) {pl.Trainer};
    % \node[class] (8) at (3.5, 7) {Priors};
    \draw [->] (sampler) -- (mcmc-module);
    \draw [->] (model) edge (mcmc-module);
    \draw [->] (model) edge (vi-module);
    \draw [->] (model) edge (sgd-module);
    \draw [->] (conv) edge (mcmc-module);
    \draw [->] (conv) edge (vi-module);
    \draw [->] (conv) edge[dashed] (sgd-module);
    \draw [->] (dataset) -- (data-module);
    \draw [->] (data-module) edge[out=-90, in=90] (trainer);
    \draw [->] (inf-modules) edge[out=-90, in=90] (trainer);
\end{tikzpicture}

    \caption{Overall architecture of inference implementation. }
    \label{fig:sw-arch}
\end{figure}

\section{Models}
The models used in this project are defined in the \texttt{src/models} directory.
We implement these as PyTorch modules with additional methods used by the other components. 
At a high level, they define some amount of trainable parameters $\theta$ and a mapping from input $x$ to output $y=f_\theta(x)$. 
They also define an observation model through $p(y|x, \theta)$ using the probability distribution objects provided by PyTorch.

The prior distributions are defined by implementing probabilistic analogs, \texttt{BayesianModule}s to the PyTorch \texttt{Module}s, which also contains an instance of \texttt{Prior} object. 
The \texttt{Prior} object stores the prior distribution for each parameter and allows the \texttt{BayesianModule} to retrieve the prior log density. 
We can dynamically turn a non-probabilistic model into a probabilistic model by replacing any PyTorch module with trainable parameters with an instance of \texttt{BayesianModule}.
How the different modules should be converted and which priors should be applied is supplied through a \texttt{BayesianConversionConfig}.

The dynamical approach can be helpful since it simplifies the models and allows for the training of models specified elsewhere, such as pre-trained models. 
It also allows for adjusting the priors dynamically and differently based on the methods in question, 

\section{Data}
We generally define data through the PyTorch \texttt{Dataset} abstraction.
The \texttt{Dataset} class implements two methods, one for getting the total size of the dataset $N$, and another for getting extracting the $i$th observation.
For use with Lightning, these datasets are wrapped in a \texttt{pl.LightningDatamodule} object, which defines how the data should be used during training, specifying data splits, batch sizes, e.t.c.
For this project, we use a general \texttt{pl.LightningDatamodule} implementation, \texttt{DataModule} in order to reduce code duplication and better be able to ensure consistency between experiments. 

\section{Inference Modules}
The inference modules define the different means of inference and are initialized using a model object alongside any inference hyper-parameters.
The different inference methods are implemented as subclasses of \texttt{pl.LightningModule}s, and are therefore meant to be used with the \texttt{pl.Trainer} object alongside a \texttt{pl.Datamodule} for model fitting. 
There are three different inference methods implemented: regular SGD (\texttt{SGDInference}), variational inference (\texttt{VariationalInference}), and Markov chain Monte Carlo inference (\texttt{MCMCInference}).
For comparing with the probabilistic methods, the SGD inference optionally allows for using MAP estimation also using this framework.

At the core, the PyTorch Lightning framework centers around the implementation of a \texttt{training\_step()} method.
This function receives a batch of training data as an argument and returns the corresponding loss.
The Lightning framework then deals with backpropagation, optimization steps, enabling and disabling gradients, pre-fetching data, moving values across different devices, e.t.c.
Therefore, for SGD inference, it is as simple as calculating the loss and passing it along.
For the two other methods, the implementation is a bit more involved.

\subsection{Variational Inference}
Variational inference is primarily implemented through the use of yet another wrapper module, \texttt{VariationalModule}.
These modules wrap the Bayesian modules, and upon initialization, delete their parameters.
They then define a pair of variational parameters, $\mu$ and $\rho$ for each parameter deleted. 
In order to make a \texttt{forward()} pass, they must sample one or more sets of new parameters with a \texttt{sample\_parameters()} method.
This method draws a given number of samples from the variational distribution and stores them in an instance variable.

The forward pass can then consist of setting the appropriate attributes of the Bayesian modules to reference one of the sets of samples instead. 
The Bayesian module can then be used for the forward pass as usual, and a counter in the variational module is incremented afterwards to use the next sample for the next forward pass.
\begin{figure}[htbp]
    \centering
    \begin{subfigure}[b]{0.48\linewidth}
        \centering
        \caption{Basic structure of \texttt{VariationalModule}}
        % spellchecker: disable


\usetikzlibrary{shapes.multipart}
\begin{tikzpicture}[
    class/.style={rounded corners=0.3cm,minimum height=1cm,fill=blue!50,draw=black},
    other/.style={circle,draw=black},
    node distance = 1cm, 
    myrect/.style={
        draw,
        rectangle split,
        rectangle split parts=#1,
        rectangle split part align=left
        },    
    back group/.style={rounded corners, draw=black, 
                    dashed, inner xsep=0.2cm, inner ysep=0.2cm, 
                    anchor=west},
    auto
    ]
    \node[
        draw, 
        rectangle, 
        rectangle split, 
        rectangle split parts=7,
        rectangle split horizontal, 
        rectangle split part fill={
            grey,
            grey,
            brightgreen,
            white,
            white,
            white,
            white
            },
        minimum height=1cm,
        inner sep=2,
        rounded corners=0.1cm
        ] (samples) {
            \nodepart{one}
            \nodepart{two}
            \nodepart{three} \texttt{i}
            \nodepart{four}
            \nodepart{five}
            \nodepart{six}
            \nodepart{seven}
        };
    \node[class,fill=blue!50, below=1.5cm of samples] (bayesian-module) {BayesianModule};
    % \begin{scope}[on background layer]
        % \end{scope}
    \draw [->] (samples.three south) -- (bayesian-module.north -| samples.three south);
    \draw[decorate,decoration={brace, amplitude=10pt, raise=2pt}]
    (samples.one north) to node[black,midway,above=10pt] (samples-label) {$n$ samples} (samples.seven north);%
    \node (inf-modules) [back group] [fit=(bayesian-module)(samples)(samples-label),label={above:VariationalModule}]{};
    \node [other,left=of bayesian-module] (input) {$x$};
    \node [other,right=of bayesian-module] (output) {$x^\prime$};
    \draw [->,thick] (input) -- (bayesian-module);
    \draw [->,thick] (bayesian-module) -- (output);
    \draw [->,thick](bayesian-module.east) to[out=0,in=330, edge node={node [sloped,above] {\tiny \texttt{i++}}}] (samples.south east);
    \end{tikzpicture}

    \end{subfigure}
    \begin{subfigure}[b]{0.48\linewidth}
        \centering
        \caption{Implementation of \texttt{training\_step()} in \texttt{VariationalInference}.}
        % spellchecker: disable


\usetikzlibrary{shapes.multipart}
\begin{tikzpicture}
    \node[class] (model) at (0, 0) {Model};
    \node[fill=grey!50,other,below=1cm of model.south] (obs) {$\tilde{\D}$};
    \node[other,right=1.5cm of model] (loss) {$\bar{\mathcal{L}}$};

    \def\tsh{0.7cm}
    \node[minimum width=0.5cm, above=\tsh of model,anchor=center,label={above:{\tiny\texttt{training\_step()}}}] (training-step) {};

    \node[pnt] (start) at (training-step.west) {};
    \node[pnt] (end) at (training-step.east) {};
    \fill[black] (start) circle (2pt);
    \fill[black] (end) circle (2pt);
    \def\radius{1cm}

    \draw[ 
        flow,
        thick,
        postaction={decorate},
        decoration={
            markings,
            mark=at position 0.125 with {\arrow {latex}},
            mark=at position 0.375 with {\arrow {latex}},
            mark=at position 0.625 with {\arrow {latex}},
            mark=at position 0.875 with {\arrow {latex}}
        }
    ]
    (start)
    arc (90:180:\radius) node (tmp1) {}
    --
    ($(tmp1)!2.0!(tmp1 |- model)$) node[midway,anchor=center,label={left:\small{sample $n$}}] (sample-n) {}
    arc (180:270:\radius) node (tmp2) {}
    -- (tmp2 -| end) node[midway,anchor=center] (get-obs) {}
    arc (270:360:\radius) node (tmp3) {}
    -- ($(tmp3)!2.0!(tmp3 |- model)$) node[midway,anchor=center] (get-loss) {}
    arc (0:90:\radius) {}
    ;
    \def\radcirc{0.7}
    \draw[flow,thick,postaction={decorate},
        decoration={markings,
        mark=at position 0.27 with {\arrow {latex}},
        mark=at position 0.52 with {\arrow {latex}},
        mark=at position 0.77 with {\arrow {latex}}
    }]
    (get-obs.center) 
    arc (90:0:\radcirc) 
    arc (360:90:\radcirc) node[pos=0.333,anchor=north] (label) {\small repeat $n$ times}
    ;
    
    \fill[black] (sample-n) circle (2pt);
    \fill[black] (get-obs) circle (2pt);
    \fill[black] (get-loss) circle (2pt);
    
    \draw [depends] (obs) edge (model);
    \draw [depends] (model) edge (loss);
    \draw [depends] (sample-n.center) -- (model);
    


    \end{tikzpicture}

    \end{subfigure}
    \caption{Diagrams showing the implementation of variational inference.}
    \label{fig:vi-arch}
\end{figure}
If averaging over $n$ gradients, we train the model by drawing $n$ for every variational module of the model and then doing $n$ forward passes of the model, where each variational module keeps track of its samples.
Then, the loss is calculated for each forward pass, averaged and returned from \texttt{training\_step()}, letting Lightning handle the backward pass. 
See \cref{fig:vi-arch} for a diagram 

\subsection{MCMC Inference}
The Markov chain Monte Carlo implementation differs a bit compared to the two other implemented methods in that we need to do more with the gradients than perform an optimization step. 
It is probably possible to use PyTorch Lightning's \texttt{on\_after\_backward()} hook to implement the HMC methods in PyTorch Lightning, however with this implementation, we instead opt to implement the sampling algorithms more explicitly.
This tactic allows for a more straightforward implementation of the samplers, which we can also use in other contexts, like directly sampling from a density function.
The different samplers are therefore implemented as  subclasses of a \texttt{Sampler} class that each define a \texttt{next\_sample()} method independently.
After initializing each sampler, we set them up by registering an object to them, defining what distribution should be sampled from. 
This object, dubbed \texttt{Samplable}, should define the following properties: The current state as a PyTorch tensor, the shape of the state, the logarithmic proportional probability density at the current state, and the corresponding gradient.  

In the context of deep learning inference, the model parameter posterior is represented by a \texttt{ParameterPosterior} object that wraps the model object and allows for the setting of different sets of observations with an \texttt{observe()} method.
This wrapper then implements the \texttt{Samplable} interface, with the state being model parameters stacked as single one-dimensional tensor, and uses the observation model and autograd to implement the remaining methods.
\begin{figure}[htbp]
    \centering
    % spellchecker: disable

\begin{tikzpicture}
    \node[class,fill=blue!50] (model) at (0, 0) {Model};
    \node[class,align=center,fill=blue!50, right=of model] (post) {Parameter-\\Posterior};
    \node[class,fill=blue!50, below=of post] (sampler) {Sampler};
    \node[other,fill=grey!50, right=3.2cm of post.center] (obs) {$\tilde{\D}$};
    \node[other,fill=red!50, right=3.2cm of sampler.center] (state) {$\theta^\ast$};
    \draw [depends] (model) -- (post);
    \draw [depends] (post) -- (sampler);
    \draw [depends] (obs) -- (post);
    \draw [depends] (sampler) -- (state);

    
    \def\yoffset{1cm}

    \node[minimum width=0, label={above:{\tiny\texttt{training\_step()}}}] 
    (start) at ($(obs)+(-0.9cm,\yoffset)$)
    {}
    ;
    
    \draw[
        thick,
        postaction={decorate},
        decoration = {
            markings,
            mark=at position 0.15 with {\arrow {latex}},
            mark=at position 0.50 with {\arrow {latex}},
            mark=at position 0.9 with {\arrow {latex}}
        }
    ] (start.center) 
    -- (obs -| start) node[inner sep=0, label={above left:{\tiny\texttt{observe()}}}] (observe) {}
    -- (state -| start) node[inner sep=0, label={above left:{\tiny\texttt{next\_sample()}}}] (next-sample) {}
    -- ($(state -| start)+(0,-\yoffset)$) node (end) {};
    ;
%     \draw[thick]
%         (start)
%         arc (90:180:\radius) node (tmp1) {}
%         -- ($(tmp1)!2.0!(tmp1 |- mid)$)
%         arc (180:270:\radius) node (tmp2) {}
%         -- (tmp2 -| end) 
%         arc (270:360:\radius) 
%         % -- ($tmp1!2.0!((obs -| tmp1)!0.5!(state -| tmp1))$)
%         ;
    % \path[name path=flow] ($(obs)+(-0.8cm,1cm)$)--($(state)+(-0.8cm,-1cm)$); 
    % \path[name path=observe] (obs)--(post); 
    % \path[name path=next] (sampler)--(state); 
    % \fill [black, name intersections={of=flow and observe}] (intersection-1) circle (2pt);
    % \fill [black, name intersections={of=flow and next}] (intersection-1) circle (2pt);
    % \path ($(obs)+(0.8cm,1cm)$) edge[draw=none] node[sloped, anchor=north] {\tiny\texttt{training\_step()}} ($(state)+(0.8cm,-1cm)$);
    % \node (a) at ($(obs)+(-0.8cm,1cm)$) {};
    % \node (b) at ($(state)+(-0.8cm,-1cm)$) {};
    \fill[black] (start) circle (2pt);
    \fill[black] (observe) circle (2pt);
    \fill[black] (next-sample) circle (2pt);
    \fill[black] (end) circle (2pt);
    % \node[red,scale=3] at (intersection of  post--obs and a--b{.};
    \end{tikzpicture}

    \caption{Basic diagram of \texttt{MCMCInference} implementation of \texttt{training\_step()} for general MCMC inference.}
    \label{fig:mcmc-arch}
\end{figure}
The MCMC inference module is thus responsible for setting up the \texttt{Samplable} wrapper with each batch of observations with \texttt{observe()}, and stepping the sampler with \texttt{next\_sample()} method. 
A diagram of the procedure can be seen in \cref{fig:mcmc-arch}.
In a particular case, when the batch size is equal to the number of elements in the whole dataset, this allows for sampling in a non-batched manner, such as with proper HMC.
This sampling strategy is not the most efficient; however, it allows for easy comparison of the different methods.

Furthermore, the module defines the additional hyperparameters, such as how many steps should be performed per sample and a burn-in period.
We define the strategy of which and how many samples we keep through the use of a \texttt{SampleContainer} object.

\section{Other frameworks}

While independently defining the different components allows for great flexibility, it can make it cumbersome to configure and instantiate experiments, as objects may need other instantiated objects to be instantiated themselves.
As the different inference methods also need different types of objects and hyperparameters, parameterizing and documenting different configurations can become unmanageable and prone to errors.

To make experimentation as painless as possible, we use the Hydra \cite{yadan_hydra_2019} framework to manage the configuration for us.  
This framework allows us to specify the different configurations in terms of the \texttt{.yaml} files found in the \texttt{conf/} directory.
The configured values can be used directly in the Python code but can also be used for object instantiation using the \texttt{\_target\_} field.
Since Hydra also supports nested instantiation.
We can instantiate just about every component of the implementation dynamically, alleviating the mentioned complexity is somewhat.

This also results in the experiments being carried out using only a few scripts: \texttt{scripts/inference.py}, \texttt{scripts/sweep.py} and \texttt{scripts/sample.py}.

With this also comes many convenient extra features such as overriding every just about anything using the command line, launching several combinations of configurations at once, and automatic management of output directory for every run.

We also use the hyperparameter optimization library Optuna \cite{akiba_optuna_2019} for automating model fitting.
This framework is also relatively easy to implement by defining the hyperparameters of interest in the corresponding configuration fields, each specified with a search space. 
Hyperparameters are then suggested by Optuna's API and subsequently injected into the main configuration. 
An example can be seen in the \texttt{scripts/sweep.py} script.
