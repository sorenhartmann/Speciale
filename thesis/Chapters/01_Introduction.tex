\chapter{Introduction}
Modern deep learning models are incredible.
Following our increase in computational resources, we have been able to construct and train ever more powerful and expressive models.
Combined with the use of massive datasets, these model are, for better or for worse, changing the world around us. 
Despite this, as the models become more complex, it also becomes more difficult to reason about how the models are actually interpreting and reasoning about the data they are fed.

For instance, we see that as models are achieving better performance, they also become increasingly poorly calibrated\cite{guo_calibration_2017}, making us question how much trust we should put in the. 
We also see examples of certain adversarial attacks, for which even powerful models completely break down CITE(https://arxiv.org/abs/1906.00204).
This warrants consideration, especially in areas where robustness of models are paramount, be that medical applications or self driving cars. 
This naturally leads us to the question of what we can do in order to achieve more robust models.

Most inference problems can generally be framed in terms of figuring out the conditional probabilistic model of some output targets given some input data.
Typically, this model is parameterized by some parameters, which are often unknown.
Training the model then usually consists of figuring out maximum likelihood estimates for these parameters, usually seen as an optimization problem.
In Bayesian statistics, instead of relying on a single point estimate of the unknown parameters, we instead seek a posterior distribution for the parameters given our dataset.
Using this method, we can generally achieve more robust results compared to just using point estimates. 
However, if we are to apply this method to deep learning models, we are faced with some difficulties due to the large amount of data and high number of parameters.

Bayesian Neural Networks has been the subject of some interest, and several papers over the recent years. 
In \cite{chen_stochastic_2014} the Hamiltonian Monte Carlo algorithm for Bayesian inference is adapted to use in a batched learning setting. 
\cite{blundell_weight_2015} a variational distribution is used for approximating the posterior, allowing for an optimization based approach.
More recently, \cite{wilson_bayesian_2019} proposes using SGD weight iterates to approximate the posterior.

This project serves is an exploration into implementing and applying the ideas from the former two papers.
First, we will walk through and describe the two approaches, including the mathematical results underpinning them.
The methods will then be compared to each other and a baseline.

We will discuss different practical considerations, and also explore to what degree Bayesian methods can be implemented as a drop in replacement for regular SGD based algorithms.
