\chapter{Experiments}

In the following chapters we discuss a series of experiments relating to the methods discussed above. 
We first discuss some experiments relating to improving the SGHMC through better estimation of the variance $\hat{V}$, and then perform some experiments related to the applicability of the methods in the context of deep learning, compared to more conventional methods.

\section{Estimating the variance}

In the SGHMC paper, they propose estimating the variance introduced through batching as $\hat{V}=0$, and then relying on the known simulated noise to make the inaccuracy of this estimate irrelevant. 
However, it seem to be worth investigating whether we could improve performance by providing some sort of estimate. 
This is also mentioned in the paper, however they do not seem explore the idea any further. 

Since we parameterized the upper bound $C$ with $\alpha$, we may end up with estimated noise larger than the upper bound.
Adjusting the upper bound $C$ to accommodate this, would also require adjusting the $\alpha$ parameter accordingly.
Since the $\alpha$ corresponds to the momentum decay of the algorithm, we may end up with very unstable behavior, especially for $\alpha > 1$. 
\todo[inline]{Sammenhæng mellem beta/V/B/D... skal nok rettes igennem fra teoriafdelingen af...}
Instead, we keep $\alpha=CM^{-1}$ fixed, and adjust the mass matrix $M$, implicitly adjusting $C$ as well.
With $M = M_0 W$ were $W$ is a diagonal matrix of weights, we can then, depending on the variance estimate, rescale the mass such that the upper bound $\alpha$ is sufficiently above the estimate $\hat\beta$, eg. by some factor $m_{\text{est}}$.
If we let the learning rate be given as $\eta_0$  for the unscaled mass matrix with $W = I$, then the learning rate for the scaled system can be found as $\eta = \epsilon^2 M^{-1} = \epsilon^2 M_0^{-1}W^{-1} = \eta_0 W^{-1}$, and
\begin{align}
    m_{\text{est}} \cdot \hat{\beta}  &\leq \alpha \Leftrightarrow\\ 
    m_{\text{est}} \cdot \frac{1}{2} \hat V \eta   &\leq \alpha \Leftrightarrow\\ 
    m_{\text{est}} \cdot \frac{1}{2} \hat V \eta_0 W^{-1}  &\leq \alpha \Leftrightarrow\\ 
    m_{\text{est}} \cdot \frac{1}{2\alpha} \hat V \eta_0   &\leq W \impliedby \\
    W_{i,i} &= \begin{cases}
        1 & \frac{1}{2\alpha}m_{\text{est}} \eta_0 \hat{V}_{i,i} < 1 \\
        \frac{1}{2\alpha}m_{\text{est}} \eta_0 \hat{V}_{i,i} & \text{otherwise}
    \end{cases}
\end{align}

The sampler then every once in a while, eg. every 50 samples, rescales the system based on the variance estimates.
If the variance estimates $\hat \beta$ happen to be greater than $\alpha$ in between the mass rescaling, the estimate is simply clamped at the upper bound $\alpha$, and no additional noise is generated.  

This method of rescaling the mass matrix is in a sense similar to the approach outlined in \cite{wenzel_how_2020}, where they also include a step calculating a preconditioning matrix, ie.
They found that using the preconditioning step generally improved performance.

There are multiple ways to go about obtaining a variance estimate.
We could try to calculate a running variance statistic using the within batch sample variance of the gradients:
\begin{equation}
    \frac{1}{|\tilde{\D}|-1}\sum_{(x_i,y_i)=\tilde{\D}} (\nabla U(\theta; x_i, y_i) - \nabla \bar{U}(\theta))^2
\end{equation}
and then scale the variance up to the fit the batch size. 
The main issue with this approach is that calculating the gradient for each observation individually within a batch is not really computationally sound.
While it may be possible to calculate the statistic above using autograd without doing a backward pass for each observation, it is trivial to do so since autograd only allows for calculating gradients of scalars. 

Instead we look at three other approaches. 
The first approach is to run through a few training batches at the start of each epoch, calculating the gradient for each batch keeping the parameters constant. 
This estimate can then be used for the remaining epoch. 
The two other approaches instead keeps track of running statistics. 
The first uses exponentially decaying running estimates of the first to moments, $m_t, v_t$ as in the ADAM algorithm \cite{kingma_adam_2017}.
Since the ADAM algorithm doesn't center the second moment estimate, the squared mean is subtracted, so that the variance is estimated as $v_t - m_t^2$.
Finally, we also try to estimate the variance as with the ADAM algorithm, however using the first moment to center the second moment estimate. 

In order to test these three different approach, they are applying to the simulated example from the previous section.
Every 10 sample, the variance of the gradient is estimated using 100 random batched from the training data set.
This estimate is then compared to the estimate.
The estimation provided by the estimators compared to the observed variance can be seen in \cref{fig:est_variances_simulated}.
\begin{figure}[htbp]
    \centering
    \includegraphics[width=0.9\textwidth]{Figures/simulated_sghmc_gradient_variance_estimations.pdf}
    \caption{Estimated variance compared to observed variances}
    \label{fig:est_variances_simulated}
\end{figure}
\begin{table}[htbp]
    \centering
    \begin{tabular}{lrrrr}
\toprule
parameter &    0 &    1 &    2 &    3 \\
variance\_estimator   &      &      &      &      \\
\midrule
AdamEstimator        & 0.65 & 2.03 & 1.67 & 2.98 \\
ConstantEstimator    & 1.00 & 1.00 & 1.00 & 1.00 \\
ExpWeightedEstimator & 0.30 & 0.41 & 0.42 & 0.49 \\
InterBatchEstimator  & 1.00 & 0.98 & 0.98 & 0.94 \\
\bottomrule
\end{tabular}

    \caption{Relative errors for the different estimation schemes,}
\end{table}
Note that these values are plot using a logarithmic scale, so the estimates may seem better than they actually are, especially with when it comes to overestimating. 
It may still be that they are better than simply setting $\hat\beta=0$ in terms of estimating the posterior correctly.
In \cref{fig:simualted_var_est_joint_comp}, the sampled marginal distribution of $a_1$ and $a_3$ are shown again, now compared SGHMC with and without a variance estimator.
\begin{figure}[htbp]
    \centering
    \begin{subfigure}[t]{0.4\textwidth}
        \centering
        \includegraphics[width=\textwidth]{Figures/simulated_joint_SGHMC_5.pdf} 
        \caption{SGHMC with with gradient variance estimated as $\hat{V}=0$.}
    \end{subfigure}
    \begin{subfigure}[t]{0.4\textwidth}
        \centering
        \includegraphics[width=\textwidth]{Figures/simulated_joint_SGHMCWithVarianceEstimator_5.pdf} 
        \caption{SGHMC with using the exponentially scaled estimator of gradient variance}
    \end{subfigure}
    \caption{Joint distribution of samples for $a_1$ and $a_3$ for the polynomial regression example for HMC and SGHMC for different batch sizes, with the actual posterior density also shown.}
    \label{fig:simualted_var_est_joint_comp}
\end{figure}
In this case, it does indeed seem like estimating the variance is helping the sampler.
This may come down to the approach with estimated variance is able to adjust the step size down, lowering discretization and batch error compared to SGHMC without an estimator. 

\subsection{Marginal distribution of momentum}

In \cite{wenzel_how_2020}, they propose using the marginal distribution of the momentum as a test of the sampling assumptions. 
Per the dynamical system, if the system is simulated exactly, the marginal distribution of $r$ should be $\mathcal{N}(0, M)$. 
We should therefore have that $r M^{-1/2} \sim \mathcal{N}(0, 1)$. 
Applying the inner product, $r^T M r$ should therefore follow a $\chi^2(d)$ distribution, where $d$ is the length of $r$. 
If we look at some $c$ quantile of the $\chi^2(d)$ distribution, we would expect a fraction $c$ of the samples for $r^T M r$ to land within that particular quantile. 
To this end, let $\hat T_{0.99}$ be the fraction of samples for some set of $d$ parameters, that fall within a $0.99$ quantile of the $\chi^2(d)$ distribution. 
We should therefore expect $\hat T_{0.99}=0.99$ given a perfectly simulated system.
This allows us to test the correctness of our sampler, even if we don't know the true posterior, which is going to be the case in the following experiments.

Applying this statistic, we can see whether the estimation of the variance helps with regard to the correctness of the simulation. 
Returning back to our simulated polynomial example, in the previous experiment, the momentum were resampled every $10$ epochs, to make to comparison to regular HMC as clear as possible. 
However, if we resample the momentum from the marginal distribution too often, the $\hat{T}_{0.99}$ measure would obviously be less informative with regard to the influence of discretization and batching error. 
The momentum is therefore instead resampled every $1000$ steps, with a slightly lower learning rate of $\eta=4 \cdot 10^{-5}$.
The lower learning rate is used since otherwise, the SGHMC sampler with $\hat{\beta}=0$ becomes unstable. 
\begin{figure}[htb]
    \centering
    \includegraphics[width=0.9\textwidth]{Figures/temperature_sum_chi2_comp.pdf}
    \caption{<caption>}
    \label{fig:temperature_sum_chi2_comp}
\end{figure}
\begin{table}[htb]
    \centering
    \begin{tabular}{lrr}
\toprule
{} &    frac\_in\_99 &        pm \\
parameter & \multicolumn{2}{l}{linear.weight} \\
variance\_estimator   &               &           \\
\midrule
AdamEstimator        &      0.992099 &  0.001363 \\
ConstantEstimator    &      0.835864 &  0.005704 \\
ExpWeightedEstimator &      0.996358 &  0.000928 \\
InterBatchEstimator  &      0.995617 &  0.001017 \\
\bottomrule
\end{tabular}

    \caption{<caption>}
    \label{tbl:temp_99}
\end{table}
The resulting distribution of $r^T M r$ can be seen in \cref{fig:temperature_sum_chi2_comp}, compared against the expected $\chi^2$ distribution. 
We see that the while the samplers with estimated variance doesn't fit the distributional assumptions, they are all much closer than the for the sampler with $\hat \beta = 0$.
The $\hat T_{0.99}$ statistic have also been calculated in \cref{tbl:temp_99}, alongside a confidence interval based on the binomial distribution.

It seem like the samplers with estimated variance are all underdispersing slightly, compared to the actual posterior. with  which would make sense if we are overestimating variance.
Looking at the PLOT, the estimation margin estimates also seems relatively uniformly distributed on the log scale, which would mean that we are generally overestimating with larger magnitude. 
Since we are overestimating the noise from batching, this would lead to us not adding sufficient noise to the system, leading to underdispersion.

\section{MNIST}
In this section, we will explore the performance of the different methods on the MNIST dataset for classification. 
This dataset is often used to benchmark and compare different methods and implementations. 
In \cite{chen_stochastic_2014} they also compare against MNIST, using a small neural network with a single hidden layer of 100 units.
For this model, they demonstrate comparable, and even superior performance compared to SGD methods using weight regularization.
The authors of \cite{blundell_weight_2015} also compare with MNIST, only they use much larger networks with two hidden layers of size 400, 800 and 1200. 
They also compare against regular SGD, also using dropout during training. showing slightly better test performance of VI.

In this project, we compare all these methods using a regular feed forward neural network with two hidden layers of size 800 and using ReLU as activation function. 
The different algorithms all have different hyperparameters, some more than others.
As an effort to keep the comparison fair between the algorithms, the hyperparameter optimization framework Optuna \cite{akiba_optuna_2019} is used. 
This framework treats the choice of hyperparameters as a black box optimization problem, trying to figure out what choices of parameters leads to the best validation error.
\todo{optimizers}

For each algorithm, we spend 24 hours to figure out sensible values for the various hyperparameters over 1000 epochs. 
Since the variational method may be a lot slower than the other methods if using a lot of samples for estimating the ELBO gradient, we impose a 2 hour time limit for each training run. 
The validation curve corresponding to the best run for each algorithm can be seen in \cref{fig:mnist-best-val-curves}.
\begin{figure}[htbp]
    \centering
    \includegraphics[width=.9\linewidth]{Figures/mnist-best-runs-val-curves.pdf}
    \caption{Validation curves for best run found during hyperparameter search.}
    \label{fig:mnist-best-val-curves}
\end{figure}

Tables documenting the choice of parameters and related validation error can be seen in \cref{apx:mnist-sweep} 
For each algorithm, the parameters are then set based on the top few results and the models are trained a final time.
Seeing as some of the algorithms tended to overfit, the best model during training according to validation error is chosen for the non-MCMC methods.
The variational algorithm is given some more time to finish training, to see whether the performance increases if we let the model train for all 1000 epochs. 

The resulting testing errors for each algorithm can be seen in \cref{tab:mnist-test-err}.
\begin{table}[htbp]
    \centering
    \begin{tabular}{ll}
\toprule
{} & Test error incl. 95\% CI \\
\midrule
SGD (MAP)              &      $0.0175 \pm 0.0026$ \\
SGD (dropout)          &      $0.0143 \pm 0.0023$ \\
SGHMC                  &      $0.0173 \pm 0.0026$ \\
SGHMC (with var. est.) &      $0.0158 \pm 0.0024$ \\
VI                     &       $0.0243 \pm 0.003$ \\
\bottomrule
\end{tabular}

    \caption{Testing errors for MNIST for different inference algorithm.}
    \label{tab:mnist-test-err}
\end{table}
We see that out of the methods attempted, regular SGD with dropout wins out. 
Interestingly, both SGHMC methods seem to perform a significantly better than suggested by the validation error curves. 
Since we used the validation set for tuning our parameters, we would in fact expect the opposite to be true. 
Anyhow, both the validation and test error seems to indicate that the SGHMC using gradient variance estimation improves the predictive performance of the models.
\todo{vi også med kl weightnign}

not entriely fair comparison... downsample mnist <-

% \subsection{Downsampling MCMC Samples}
% For this experiment, we keep parameter samples for every epoch after the initial burn in period.
For larger models, this may not be entirely feasible.
It may also be worth exploring to what degree a larger ensemble of models affects the test error.
In

% \begin{figure}[htbp]
%     \centering
%     \includegraphics[width=\linewidth]{Figures/mnist-downsampling.pdf}  
%     \caption{Test error of }
%     \label{<label>}
% \end{figure}
\subsection{Model confidence}

As mentioned in the introduction, overconfident models are a common phenomenon within modern deep learning, especially for deeper models. 
A key motivator for experimenting with Bayesian methods has been to try and get a more robust and reliable estimate.
As in CITE we use the \emph{expected calibration error} (ECE) for benchmarking the uncertainty of the different methods.
For some set of predictions, the \emph{confidence} is defined as the estimated probability $\hat{p}$ of belonging to the estimated class.
The ECE statistic compares the confidence of the model to the observed accuracies.
This is done by first grouping the test observations into $M$ bins of size $1/M$, according to the confidence. 

Then, the accuracy of each bin $B_m$, can be calculated as:
\begin{align}
    \acc(B_m) = \frac{1}{|B_m|}\sum_{i\in B_m}\bm{1}(\hat{y}_i = y_i)
\end{align}
Likewise, we calculate the average model confidence across the bin
\begin{align}
    \conf(B_m) = \frac{1}{|B_m|}\sum_{i\in B_m} \hat{p}_i.
\end{align}
Now, if the confidence of the model is to be trusted, we would expect $\acc{B_m} = \conf{B_m}$.
These measures thus gives us a way to benchmark the uncertainty of the model.
Further, we also calculate the ECE as the expected absolute difference between the accuracy and the confidence weighted according to the size of each bin
\begin{align}
    \text{ECE} = \sum_{m=1}^{M} \frac{|B_m|}{n}|\acc(B_m)-\conf(B_m)|.
\end{align}
This gives us a single statistic to compare the inference methods across.
In \cref{tab:mnist-ece}, the ECE for the test set can be seen for each of the inference methods. 
\todo{comment on result}
\begin{table}[htbp]
    \centering
    \begin{tabular}{lr}
\toprule
{} &   ECE \\
\midrule
SGD (dropout)          & 1.39\% \\
SGD (MAP)              & 1.00\% \\
VI (exp. KL weight)    & 0.77\% \\
SGHMC                  & 3.04\% \\
SGHMC (with var. est.) & 1.55\% \\
\bottomrule
\end{tabular}

    \caption{<caption>}
    \label{tab:mnist-ece}
\end{table}
\begin{figure}[htbp]
    \centering
    \includegraphics[width=\linewidth]{Figures/mnist-calibration.pdf}
    \caption{<caption>}
    \label{<label>}
\end{figure}
Interestingly, we see that the models trained with MCMC are both overconfident in their predictions.  
Cifar bliver sgu nok sjovere. -> check assumptions
\subsection{Checking SGHMC Assumptions}
As with the polynomial model, we can use the marginal distribution of the momentum variables $r$ to examine how well the sampled distribution resembles the correct distribution. 
\begin{figure}[htbp]
    \centering
    \includegraphics[width=\linewidth]{Figures/mnist-temperatures.pdf}
    \caption{Temperature distributions for different parameter groups in the model during SGHMC training.}
    \label{fig:mnist-temperatures}
\end{figure}
We see that samples for neither sampling method really matches the correct marginal distribution for the momentum variables. 
It therefore seems like the sampling methods breaks down for many parameters compared to the simple examples explored in the previous experiments.

It is worth noting, that we optimized the choice of hyperparameters with respect to the validation error, which does not necessarily imply correctness of the samples from the posterior. 


\todo{Having the probabilistic model is ill-posed -> to discussion?}


\section{Convolutional Neural Net}
While the combination of MNIST and feed forward neural networks are great for getting


\begin{table}[htbp]
    \centering
    \begin{tabular}{ll}
\toprule
{} & Test error incl. 95\% CI \\
\midrule
SGD (MAP)     &      $37.57 \pm 0.95~\%$ \\
SGD (dropout) &      $31.86 \pm 0.91~\%$ \\
VI            &      $29.59 \pm 0.89~\%$ \\
\bottomrule
\end{tabular}

    \caption{<caption>}
    \label{<label>}
\end{table}

\begin{figure}[htbp]
    \centering
    \includegraphics[width=\linewidth]{Figures/cifar10-small-calibration.pdf}
    \caption{<caption>}
    \label{<label>}
\end{figure}
\begin{table}[htbp]
    \centering
    \begin{tabular}{lc}
\toprule
                Method &             ECE \\
\midrule
         SGD (dropout) &         21.87\% \\
             SGD (MAP) &         18.47\% \\
           \textbf{VI} & \textbf{3.62\%} \\
                 SGHMC &          7.55\% \\
SGHMC (with var. est.) &          6.54\% \\
\bottomrule
\end{tabular}

    \caption{<caption>}
    \label{<label>}
\end{table}

\subsection{CIFAR10}

\subsection{Densenet}


