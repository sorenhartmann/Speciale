% spellchecker: disable


\usetikzlibrary{shapes.multipart}
\begin{tikzpicture}[
    class/.style={rounded corners=0.3cm,minimum height=1cm,fill=blue!50,draw=black},
    other/.style={circle,draw=black},
    node distance = 1cm, 
    myrect/.style={
        draw,
        rectangle split,
        rectangle split parts=#1,
        rectangle split part align=left
        },    
    back group/.style={rounded corners, draw=black, 
                    dashed, inner xsep=0.2cm, inner ysep=0.2cm, 
                    anchor=west},
    auto
    ]
    \node[
        draw, 
        rectangle, 
        rectangle split, 
        rectangle split parts=7,
        rectangle split horizontal, 
        rectangle split part fill={
            grey,
            grey,
            brightgreen,
            white,
            white,
            white,
            white
            },
        minimum height=1cm,
        inner sep=2,
        rounded corners=0.1cm
        ] (samples) {
            \nodepart{one}
            \nodepart{two}
            \nodepart{three} \texttt{i}
            \nodepart{four}
            \nodepart{five}
            \nodepart{six}
            \nodepart{seven}
        };
    \node[class,fill=blue!50, below=1.5cm of samples] (bayesian-module) {BayesianModule};
    % \begin{scope}[on background layer]
        % \end{scope}
    \draw [->] (samples.three south) -- (bayesian-module.north -| samples.three south);
    \draw[decorate,decoration={brace, amplitude=10pt, raise=2pt}]
    (samples.one north) to node[black,midway,above=10pt] (samples-label) {$n$ samples} (samples.seven north);%
    \node (inf-modules) [back group] [fit=(bayesian-module)(samples)(samples-label),label={above:VariationalModule}]{};
    \node [other,left=of bayesian-module] (input) {$x$};
    \node [other,right=of bayesian-module] (output) {$x^\prime$};
    \draw [->,thick] (input) -- (bayesian-module);
    \draw [->,thick] (bayesian-module) -- (output);
    \draw [->,thick](bayesian-module.east) to[out=0,in=330, edge node={node [sloped,above] {\tiny \texttt{i++}}}] (samples.south east);
    \end{tikzpicture}
