%\usepackage{microtype}      % better looking text
%\usepackage[utf8]{inputenc}
\usepackage{fontspec}       % Package for custom fonts
\usepackage[]{geometry}     % Package for changing page margins (before fancyhdr) 
\usepackage{fancyhdr}       % Package to change header and footer
\usepackage{parskip}        % Package to tweak paragraph skipping (instead of indents a small skip is added after every paragraph)
\usepackage{titlesec}
\usepackage{tikz}           % Package for drawing
\usepackage{pgfplots}       % Package for creating graphs and charts
\usepackage{xcolor}         % Package for defining DTU colours to be used
\usepackage{amsmath}        % For aligning equations among other
\usepackage{siunitx}        % SI units
\usepackage{listings}       % Package for inserting code, (before cleveref)
\PassOptionsToPackage{hyphens}{url} % Ability to line break urls at hyphens
\usepackage{hyperref}       % Package for cross referencing (also loads url package)
\usepackage{cleveref}       % improved cross referencing
\usepackage{textcomp}       % \textdegree = °C and other useful symbols
\usepackage[english]{babel} % localisation 
\usepackage{caption}        % better captions
\usepackage{subcaption}     % for subfigures
\usepackage{csquotes}       % For biblatex with babel
\usepackage[backend=biber,style=numeric,sorting=none]{biblatex} % Package for bibliography (citing)
\bibliography{bibliography.bib}
\usepackage{tabularx}       % for ability to adjust column spacing in tabular better
\usepackage{booktabs}       % for better tables
\usepackage{float}          % floating figures in correct places
\usepackage{calc}           % Adds ability for latex to calculate (3pt+2pt) 
% \usepackage[printwatermark=false]{xwatermark} % Package for wartermark. Toggle printwatermark true or false to include or remove the watermar
\usepackage{blindtext}

\usepackage{silence}
\usepackage{lmodern}
% \usepackage{charter}
% \WarningFilter{mathdesign}{Font shape}
% \usepackage[charter,cal=cmcal]{mathdesign}
% \usepackage[charter]{mathdesign}

\usepackage{algorithm}
\usepackage{algpseudocode}
\usepackage{bm}
\usepackage{todonotes}
\usepackage{physics}

\usetikzlibrary{bayesnet}

\newcommand{\question}[1]{\todo[inline,color=green!20!white]{#1}}